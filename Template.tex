\documentclass{article}
\usepackage{minted}
\usepackage{cleveref}
\usepackage{graphicx}
\usepackage{listings}
\usepackage{courier}
\usepackage[backend=biber,citestyle=authoryear,url=true,backref=bibtex,bibstyle=apa,useprefix=true]{biblatex}
\setminted[python]{fontfamily=courier}

\addbibresource{bibliography.bib}
\begin{document}
	\section{Source code}\label{sec:code} % (fold)
	The source code of this application, in its entirety can be found below, split up as follows: 
	\begin{itemize}
		\item Main source code 
		\item Facilities
		\item Exceptions
		\item Comparators
	\end{itemize}

	\subsection{Main source code}\label{sub:main_source_code} % (fold)
	\textit{MedicalConsole.java}
	\inputminted{java}{src/main/java/com/yvesstraten/medicalconsole/MedicalConsole.java}

	\pagebreak

	\textit{Format.java}
	\inputminted{java}{src/main/java/com/yvesstraten/medicalconsole/Format.java}

	\textit{Input.java}
	\inputminted{java}{src/main/java/com/yvesstraten/medicalconsole/Input.java}

	\textit{ArrayListSet.java}
	\inputminted{java}{./src/main/java/com/yvesstraten/medicalconsole/ArrayListSet.java}

	\textit{Editable.java}
	\inputminted{java}{src/main/java/com/yvesstraten/medicalconsole/Editable.java}

	\pagebreak

	\textit{HealthService.java}
	\inputminted{java}{src/main/java/com/yvesstraten/medicalconsole/HealthService.java}

	\textit{IdGenerator.java}
	\inputminted{java}{src/main/java/com/yvesstraten/medicalconsole/IdGenerator.java}
	% subsection Main source code (end)

	\subsection{Facilities}\label{sub:facilities} % (fold)
	\textit{Clinic.java}
	\inputminted{java}{src/main/java/com/yvesstraten/medicalconsole/facilities/Clinic.java}

	\pagebreak

	\textit{Hospital.java}
	\inputminted{java}{src/main/java/com/yvesstraten/medicalconsole/facilities/Hospital.java}

	\textit{MedicalFacility.java}
	\inputminted{java}{src/main/java/com/yvesstraten/medicalconsole/facilities/MedicalFacility.java}

	\textit{Procedure.java}
	\inputminted{java}{src/main/java/com/yvesstraten/medicalconsole/facilities/Procedure.java}

	\textit{Patient.java}
	\inputminted{java}{src/main/java/com/yvesstraten/medicalconsole/Patient.java}
	% subsection Facilities (end)

	\pagebreak
	
	\subsection{Comparators}\label{sub:comparators} % (fold)
	\textit{MedicalFacilitiesComparators.java}
	\inputminted{java}{src/main/java/com/yvesstraten/medicalconsole/comparators/MedicalFacilitiesComparators.java}

	\textit{PatientComparators.java}
	\inputminted{java}{src/main/java/com/yvesstraten/medicalconsole/comparators/PatientComparators.java}

	\textit{ProcedureComparators.java}
	\inputminted{java}{src/main/java/com/yvesstraten/medicalconsole/comparators/ProcedureComparators.java}
	% subsection Comparators (end)

  \newpage

	\subsection{Exceptions}\label{sub:exceptions} % (fold)
	\textit{ClassIsNotEditableException.java}
	\inputminted{java}{src/main/java/com/yvesstraten/medicalconsole/exceptions/ClassIsNotEditableException.java}

	\textit{InvalidOptionException.java}
	\inputminted{java}{src/main/java/com/yvesstraten/medicalconsole/exceptions/InvalidOptionException.java}

	\textit{InvalidYesNoException.java}
	\inputminted{java}{src/main/java/com/yvesstraten/medicalconsole/exceptions/InvalidYesNoException.java}

	\textit{NegativeNumberException.java}
	\inputminted{java}{src/main/java/com/yvesstraten/medicalconsole/exceptions/NegativeNumberException.java}

	\textit{NoHospitalsAvailableException.java}
	\inputminted{java}{src/main/java/com/yvesstraten/medicalconsole/exceptions/NoHospitalsAvailableException.java}

	\textit{WrongHospitalException.java}
	\inputminted{java}{src/main/java/com/yvesstraten/medicalconsole/exceptions/WrongHospitalException.java}
	% subsection Exceptions (end)

	\section{Unit tests}\label{sec:unit_tests} % (fold)
	For this assignment, unit tests written in JUNIT 5 \textcite{Junit5} were used extensively. Each unit test is show as follows:
	
	\textit{AddTests.java} 
	\inputminted{java}{src/test/java/com/yvesstraten/medicalconsole/tests/AddTests.java}

	\textit{ClinicTests.java} 
	\inputminted{java}{src/test/java/com/yvesstraten/medicalconsole/tests/ClinicTests.java}

	\textit{DeleteTests.java} 
	\inputminted{java}{src/test/java/com/yvesstraten/medicalconsole/tests/DeleteTests.java}

	\textit{EditTests.java} 
	\inputminted{java}{src/test/java/com/yvesstraten/medicalconsole/tests/EditTests.java}

	\textit{FacilitiesTests.java} 
	\inputminted{java}{src/test/java/com/yvesstraten/medicalconsole/tests/FacilitiesTests.java}

	\textit{FormatTests.java} 
	\inputminted{java}{src/test/java/com/yvesstraten/medicalconsole/tests/FormatTests.java}

	\textit{HealthServiceTests.java} 
	\inputminted{java}{src/test/java/com/yvesstraten/medicalconsole/tests/HealthServiceTests.java}

	\textit{HospitalTests.java} 
	\inputminted{java}{src/test/java/com/yvesstraten/medicalconsole/tests/HospitalTests.java}

	\textit{MedicalConsoleTest.java} 
	\inputminted{java}{src/test/java/com/yvesstraten/medicalconsole/tests/MedicalConsoleTest.java}

	\textit{MedicalConsoleTests.java} 
	\inputminted{java}{src/test/java/com/yvesstraten/medicalconsole/tests/MedicalConsoleTests.java}

	\textit{PatientTests.java} 
	\inputminted{java}{src/test/java/com/yvesstraten/medicalconsole/tests/PatientTests.java}

	\textit{ProcedureTests.java} 
	\inputminted{java}{src/test/java/com/yvesstraten/medicalconsole/tests/ProcedureTests.java}

	\textit{SortingTest.java} 
	\inputminted{java}{src/test/java/com/yvesstraten/medicalconsole/tests/SortingTest.java}

	When running these tests with \textit{mvn test} the following output is shown:
	\begin{figure}
		\begin{center}
			\includegraphics[width=0.8\textwidth]{figures/Mvn_test.png}
		\end{center}
		\caption{Test output}\label{fig:}
	\end{figure}
	
	% section Unit tests (end)

	\section{Runtime output}\label{sec:runtime_output} % (fold)
	Several runtime tests have been undertaken as well to test all functions of the program. These are:
	\begin{itemize}
		\item Adding objects
		\item Listing objects
		\item Deleting objects
		\item Simulate a visit
		\item Operate on a patient
		\item Sorting objects 
		\item Edit objects
	\end{itemize}

	\subsection{Adding objects}\label{sub:adding_objects} % (fold)
	In this test, addition of objects will be tested with random misinputs to also check for the robustness of the system. The tests are undertaken in the following order: 
	\begin{itemize}
		\item Clinic 
		\item Hospital 
		\item Patient
		\item Procedure
	\end{itemize}

	\textit{Clinic}
	\begin{figure}
		\begin{center}
			\includegraphics[width=0.7\textwidth]{figures/Adding/Clinic/Clinic_01.png}
		\end{center}
		\caption{Getting to the addition menu}\label{fig:clinic_01}
	\end{figure}

	\begin{figure}
		\begin{center}
			\includegraphics[width=0.7\textwidth]{figures/Adding/Clinic/Clinic_02.png}
		\end{center}
		\caption{Adding clinic}\label{fig:clinic_02}
	\end{figure}

	\textit{Hospital}
	\begin{figure}
		\begin{center}
			\includegraphics[width=0.5\textwidth]{figures/Adding/Hospital/Hospital_01.png}
		\end{center}
		\caption{Adding a hospital}\label{fig:hospital_01}
	\end{figure}

	\newpage

	\textit{Patient}
	\begin{figure}
		\begin{center}
			\includegraphics[width=0.5\textwidth]{figures/Adding/Patient/Patient_01.png}
		\end{center}
		\caption{Adding a patient}\label{fig:patient_01}
	\end{figure}

	\textit{Procedure}
	\begin{figure}
		\begin{center}
			\includegraphics[width=0.7\textwidth]{figures/Adding/Procedure/Procedure_01.png}
		\end{center}
		\caption{Getting to the procedure menu}\label{fig:procedure_01}
	\end{figure}
	
	\begin{figure}
		\begin{center}
			\includegraphics[width=0.7\textwidth]{figures/Adding/Procedure/Procedure_02.png}
		\end{center}
		\caption{Adding a procedure}\label{fig:procedure_02}
	\end{figure}

	As you can see, objects can be freely added, and each input is checked for validity. Furthermore, each input can be repeated on error, without sending the user back to the main menu each time.
	% subsection Adding objects (end)

	\newpage 

	\subsection{Listing objects}\label{sub:listing_objects} % (fold)
	In this test, the listing of objects will be tested in the same way as addition testing. This run is done immediately after the addition of these test objects.

	\textit{Medical facilities}
	\begin{figure}
		\begin{center}
			\includegraphics[width=0.7\textwidth]{figures/Listing/Listing_Medical_01.png}
		\end{center}
		\caption{Getting to the Medical List menu}\label{fig:listing_medical_01}
	\end{figure}
	
	\begin{figure}
		\begin{center}
			\includegraphics[width=0.7\textwidth]{figures/Listing/Listing_Medical_02.png}
		\end{center}
		\caption{Listing medical facilities}\label{fig:listing_medical_02}
	\end{figure}
	
	\textit{Patients}
	\begin{figure}
		\begin{center}
			\includegraphics[width=0.7\textwidth]{figures/Listing/Listing_Patients.png}
		\end{center}
		\caption{Listing patients}\label{fig:listing_patients}
	\end{figure}
	
	\textit{Procedures}
	\begin{figure}
		\begin{center}
			\includegraphics[width=0.7\textwidth]{figures/Listing/Listing_Procedures.png}
		\end{center}
		\caption{Listing procedures}\label{fig:listing_procedures}
	\end{figure}
	% subsection Listing objects (end)

	\subsection{Deleting objects}\label{sub:deleting_objects} % (fold)
	In this test, the listing of objects will be tested in the same way as addition testing. This run is done immediately after the addition of these test objects.

	\textit{Hospital}
	\begin{figure}
		\begin{center}
			\includegraphics[width=0.95\textwidth]{figures/Deleting/Deleting_Hospital_01.png}
		\end{center}
		\caption{Getting to the facility deletion menu}\label{fig:deleting_hospital_01}
	\end{figure}
	
	\begin{figure}
		\begin{center}
			\includegraphics[width=0.95\textwidth]{figures/Deleting/Deleting_Hospital_02.png}
		\end{center}
		\caption{Deleting hospital with procedures}\label{fig:deleting_hospital_02}
	\end{figure}
	
	\textit{Clinic}
	\begin{figure}
		\begin{center}
			\includegraphics[width=0.95\textwidth]{figures/Deleting/Deleting_Clinic_02.png}
		\end{center}
		\caption{Deleting clinic}\label{fig:deleting_clinic_02}
	\end{figure}
	
	\textit{Patient}
	\begin{figure}
		\begin{center}
			\includegraphics[width=0.95\textwidth]{figures/Deleting/Deleting_Patient_01.png}
		\end{center}
		\caption{Deleting Patient}\label{fig:deleting_patient_01}
	\end{figure}
	

	
	% subsection Deleting objects (end)
	% section Runtime output (end)

	\section{Bibliography}\label{sec:bibliography} % (fold)
	\printbibliography
	% section Bibliography (end)

\end{document}
